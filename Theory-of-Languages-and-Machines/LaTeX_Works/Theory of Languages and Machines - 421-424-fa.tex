\documentclass[]{article}
\usepackage[a4paper,bindingoffset=0.2in,%
left=1in,right=1in,top=1in,bottom=1in,%
footskip=.25in]{geometry}
\usepackage{lmodern}
\usepackage{amssymb,amsmath}
\usepackage{ifxetex,ifluatex}
\usepackage{fixltx2e} % provides \textsubscript
\ifnum 0\ifxetex 1\fi\ifluatex 1\fi=0 % if pdftex
  \usepackage[T1]{fontenc}
  \usepackage[utf8]{inputenc}
\else % if luatex or xelatex
  \ifxetex
    \usepackage{mathspec}
  \else
    \usepackage{fontspec}
  \fi
  \defaultfontfeatures{Ligatures=TeX,Scale=MatchLowercase}
\fi

% use upquote if available, for straight quotes in verbatim environments
\IfFileExists{upquote.sty}{\usepackage{upquote}}{}
% use microtype if available
\IfFileExists{microtype.sty}{%
\usepackage{microtype}
\UseMicrotypeSet[protrusion]{basicmath} % disable protrusion for tt fonts
}{}
\usepackage[unicode=true]{hyperref}
\hypersetup{
            pdfborder={0 0 0},
            breaklinks=true}
\urlstyle{same}  % don't use monospace font for urls
\usepackage{graphicx,grffile}
\makeatletter
\def\maxwidth{\ifdim\Gin@nat@width>\linewidth\linewidth\else\Gin@nat@width\fi}
\def\maxheight{\ifdim\Gin@nat@height>\textheight\textheight\else\Gin@nat@height\fi}
\makeatother
% Scale images if necessary, so that they will not overflow the page
% margins by default, and it is still possible to overwrite the defaults
% using explicit options in \includegraphics[width, height, ...]{}
\setkeys{Gin}{width=\maxwidth,height=\maxheight,keepaspectratio}
\IfFileExists{parskip.sty}{%
\usepackage{parskip}
}{% else
\setlength{\parindent}{0pt}
\setlength{\parskip}{6pt plus 2pt minus 1pt}
}
\setlength{\emergencystretch}{3em}  % prevent overfull lines
\providecommand{\tightlist}{%
  \setlength{\itemsep}{0pt}\setlength{\parskip}{0pt}}
\setcounter{secnumdepth}{0}
% Redefines (sub)paragraphs to behave more like sections
\ifx\paragraph\undefined\else
\let\oldparagraph\paragraph
\renewcommand{\paragraph}[1]{\oldparagraph{#1}\mbox{}}
\fi
\ifx\subparagraph\undefined\else
\let\oldsubparagraph\subparagraph
\renewcommand{\subparagraph}[1]{\oldsubparagraph{#1}\mbox{}}
\fi

\date{}
\usepackage{xepersian}
\settextfont{BZar}

\begin{document}


\begin{center}
بسمه تعالی
\end{center}


اگر بخواهیم یک ماشین پشته ای طراحی کنیم که وقتی پشته خالی است پذیرفته شود، تابع $\lambda$ برابر بصورت زیر می باشد:

\begin{center}
\lr{$\delta$(q\textsubscript{1}, $\lambda$, z\textsubscript{0}) → (q\textsubscript{1},$\lambda$)}
\end{center}

اگر بخواهیم ماشین پشته ای طراحی کنیم که در حالت نهایی پذیرفته شود تابع $\lambda$ بصورت زیر خواهد شد:

\begin{center}
\lr{$\delta$(q\textsubscript{1}, $\lambda$, z\textsubscript{0}) → (q\textsubscript{f}, $\lambda$)}
\end{center}

11. یک ماشین پشته ای غیر قطعی طراحی کنید که زبانی را که گرامر روبرو تولید می کند تولید کند. پس از طراحی رشته \lr{abcba} را روی ماشین آزمایش کنید که پذیرفته می شود یا خیر.

حل: قوانین ضرب در حالت فرم نرمال گریباخ نمی باشند پس اولین کاری که باید انجام دهیم تبدیل آنها به فرم نرمال گریباخ می باشد.

\begin{center}
\lr{S → aSa \textbar{} bSb \textbar{} c}
\end{center}

دو قانون جدید را در نظر می گیریم
\begin{center}
\lr{C\textsubscript{a} → a , C\textsubscript{b} → b}
\end{center}

قوانین جدید بصورت زیر خواهند شد:
\begin{center}
\lr{S → aSC\textsubscript{a}}
\lr{S → bSC\textsubscript{b}}
\lr{S → c}
\lr{C\textsubscript{a} → a}
\lr{C\textsubscript{b} → b}
\end{center}

حال تمامی قوانین بصورت فرم نرمال گریباخ می باشند. حال با استفاده از قوانین بالا براحتی می توانیم ماشین پشته ای قطعی را بسازیم.
ابتدا علامت شروع که همان \lr{S} است را درون پشته قرار می دهیم با استفاده از قانون زیر
\begin{center}
\lr{$\delta$(q\textsubscript{0}, $\epsilon$, z\textsubscript{0}) → (q\textsubscript{1},Sz\textsubscript{0})}
\end{center}

سپس برای قانون \lr{S → aSC\textsubscript{a} } تابع انتقال زیر را انجام می دهیم:

\begin{center}
\lr{$\delta$(q\textsubscript{1}, a, S) → (q\textsubscript{1}, SC\textsubscript{a})}
\end{center}

برای قانون \lr{S → bSC\textsubscript{b} } تابع انتقال زیر را داریم:
\begin{center}
\lr{$\delta$(q\textsubscript{1},b,S) → (q\textsubscript{1}, SC\textsubscript{b})}
\end{center}

برای قانون \lr{ S → c} تابع انتقال زیر را داریم:

\begin{center}
\lr{$\delta$(q\textsubscript{1}, c, S) → (q\textsubscript{1}, $\lambda$)}
\end{center}

برای قانون \lr{ C\textsubscript{a} → a} تابع انتقال زیر را داریم:

\begin{center}
\lr{$\delta$(q\textsubscript{1}, a, C\textsubscript{a}) → (q\textsubscript{1},Sz\textsubscript{0})}
\end{center}

برای قانون \lr{ C\textsubscript{b} → b} تابع انتقال زیر را داریم:
\begin{center}
\lr{$\delta$(q\textsubscript{1}, b, C\textsubscript{b}) → (q\textsubscript{1}, $\lambda$)}
\end{center}

برای حالت های پایانی نیزتابع انتقال زیر را خواهیم داشت:
\begin{center}
\lr{$\delta$(q\textsubscript{1}, $\lambda$, z\textsubscript{0}) → (q\textsubscript{f},z\textsubscript{0})}
\end{center}

که با حالت پایانی پذیرفته می شود
\begin{center}
\lr{$\delta$(q\textsubscript{1}, $\lambda$, z\textsubscript{0}) → (q\textsubscript{1}, $\lambda$)}
\end{center}
که با حالت پشته خالی پذیرفته می شود

برای رشته \lr{ abcba} نیز خواهیم داشت:

\begin{flushleft}
\lr{$\delta$(q\textsubscript{0}, \underline{$\epsilon$}abcba, z\textsubscript{0}) → $\delta$(q\textsubscript{1},\emph{a}bcba, Sz\textsubscript{0}) → $\delta$(q\textsubscript{1}, a\underline{b}cba, SC\textsubscript{a}z\textsubscript{0}) → $\delta$(q\textsubscript{1}, ab\underline{c}ba, SC\textsubscript{b}C\textsubscript{a}z\textsubscript{0}) → $\delta$(q\textsubscript{1}, abc\underline{b}a, C\textsubscript{b}C\textsubscript{a}z\textsubscript{0}) → $\delta$(q\textsubscript{1}, abcb\underline{a}, C\textsubscript{a}z\textsubscript{0}) → $\delta$(q\textsubscript{1}, abcbaB, z\textsubscript{0}) → (q\textsubscript{f}, z\textsubscript{0})}
\end{flushleft}
حالت پذیرش پایانی



12. یک ماشین پشته ای قطعی طراحی کنید که معادل گرامر مستقل از متن زیر باشد:

\begin{center}
\lr{S → 0BB, B → 0S \textbar{} 1S \textbar{} 0}
\end{center}

همچنین آزمایش کنید که آیا رشته \lr{010000} در ماشین بالا پذیرفته می شود یا خیر

گرامر مستقل از متن معادل عبارت انداز

\begin{center}
\lr{S → 0BB, B → 0S \textbar{} 1S \textbar{} 0}
\end{center}

که تمامی قوانین به فرم نرمال گریباخ باشد
ابتدا علامت شروع که همان \lr{S} است را درون پشته قرار می دهیم با استفاده از قانون زیر

\begin{center}
\lr{$\delta$(q\textsubscript{0}, $\epsilon$, z\textsubscript{0}) → (q\textsubscript{1}, Sz\textsubscript{0})}
\end{center}

سپس برای قانون  \lr{S → 0BB}  تابع انتقال زیر را انجام می دهیم:

\begin{center}
	\lr{$\delta$(q\textsubscript{1}, 0, S) → (q\textsubscript{1}, BB)}
\end{center}

برای قانون \lr{ B → 0S} تابع انتقال زیر را داریم:

\begin{center}
	\lr{$\delta$(q\textsubscript{1}, 0, B) → (q\textsubscript{1}, S)}
\end{center}

برای قانون \lr{ B → 1S} تابع انتقال زیر را داریم:

\begin{center}
	\lr{$\delta$(q\textsubscript{1}, 1, B) → (q\textsubscript{1}, S)}
\end{center}

برای قانون \lr{ B → 0} تابع انتقال زیر را داریم:

\begin{center}
	\lr{$\delta$(q\textsubscript{1}, 0, B) → (q\textsubscript{1}, $\lambda$)}
\end{center}

برای حالت های پایانی نیز تابع انتقال زیر را خواهیم داشت:

\begin{center}
	\lr{$\delta$(q\textsubscript{1}, $\lambda$, z\textsubscript{0}) → (q\textsubscript{f},z\textsubscript{0})}
\end{center}
		
که با حالت پایانی پذیرفته می شود

\begin{center}
	\lr{$\delta$(q\textsubscript{1}, $\lambda$, z\textsubscript{0}) → (q\textsubscript{1}, $\lambda$)}
\end{center}
که با حالت پشته خالی پذیرفته می شود

همچنین برای رشته \lr{010000} خواهیم داشت:

\begin{flushleft}
	\lr{ (q\textsubscript{0}, \underline{$\epsilon$}10000, z\textsubscript{0}) → (q\textsubscript{1},\underline{0}10000, Sz\textsubscript{0}) → (q\textsubscript{1}, 0\underline{1}0000, BBz\textsubscript{0}) → (q\textsubscript{1}, 01\underline{0}000, SBz\textsubscript{0}) → (q\textsubscript{1}, 010\underline{0}00, BBBz\textsubscript{0}) → (q\textsubscript{1}, 0100\underline{0}0, BBz\textsubscript{0}) → (q\textsubscript{1}, 01000\underline{0}, Bz\textsubscript{0}) → (q\textsubscript{1}, 010000$\epsilon$, z\textsubscript{0}) → (q\textsubscript{f},010000$\epsilon$, z\textsubscript{0})}
\end{flushleft}
حالت پایانی پذیرش



13. نشان دهید که زبان
\begin{center}
	\lr{L = \{0\textsuperscript{n}1\textsuperscript{n} \textbar{} n \underline{\textgreater{}} 1\} $\bigcup$ \{0\textsuperscript{n}1\textsuperscript{2n} \textbar{} n \underline{\textgreater{}} 1\} }
\end{center}

که یک زبان مستقل از متن می باشد با یک ماشین پشته ای قطعی پذیرفته می شود.

حل: گرامر مستقل از متن برای زبان فوق عبارت انداز:

\begin{center}
	\lr{S → S\textsubscript{1} \textbar{} S\textsubscript{2}}
\end{center}
\begin{center}
	\lr{S\textsubscript{1} → 0S\textsubscript{1}1 \textbar{} 01}
\end{center}
\begin{center}
	\lr{S\textsubscript{2} → 0S\textsubscript{2}11 \textbar{} 011}
\end{center}

فرم نرمال گریباخ معادل آن نیز عبارت انداز:

\begin{center}
	\lr{S → 0S\textsubscript{1}A \textbar{} 0A \textbar{} 0S\textsubscript{2}A \textbar{} 0AA }
\end{center}
\begin{center}
	\lr{A → 1}
\end{center}

تابع انتقال ماشین پشته ای معادل با گرامر نیز عبارت انداز:

\begin{center}
	\lr{$\delta$(q\textsubscript{0}, $\epsilon$, z\textsubscript{0}) → (q\textsubscript{1},Sz\textsubscript{0})}
\end{center}
\begin{center}
	\lr{$\delta$(q\textsubscript{1}, 0, S) → (q\textsubscript{1}, S\textsubscript{1}A)}
\end{center}
\begin{center}
	\lr{$\delta$(q\textsubscript{1}, 0, S) → (q\textsubscript{1}, A)}
\end{center}
\begin{center}
	\lr{$\delta$(q\textsubscript{1}, 0, S) → (q\textsubscript{1}, S\textsubscript{2}A)}
\end{center}
\begin{center}
	\lr{$\delta$(q\textsubscript{1}, 0, S) → (q\textsubscript{1}, AA)}
\end{center}
\begin{center}
	\lr{$\delta$(q\textsubscript{1}, 1, A) → (q\textsubscript{1}, $\lambda$)}
\end{center}
\begin{center}
	\lr{$\delta$(q\textsubscript{1}, $\lambda$, z\textsubscript{0}) → (q\textsubscript{f}, z\textsubscript{0})}
\end{center}

با حالت پایانی پذیرفته می شود

\begin{center}
	\lr{$\delta$(q\textsubscript{1}, $\lambda$, z\textsubscript{0}) → (q\textsubscript{1}, $\lambda$)}
\end{center}
با پشته خالی پذیرفته می شود

ماشین پشته ای یک ماشین پشته ای غیر قطعی می باشد که برای ترکیب\lr{ (q\textsubscript{1}, 0, S)} چهار تابع انتقال وجود دارد.



14. گرامر مستقل از متن زیر را به یک ماشین پشته ای تبدیل کنید:

\begin{center}
	\lr{S → aAA}
\end{center}
\begin{center}
	\lr{A → aS \textbar{} bS \textbar{} a}
\end{center}

حل: گرامر در فرم نرمال گریباخ می باشد.

ابتدا علامت شروع که همان \lr{S} است را درون پشته قرار می دهیم با استفاده از قانون زیر

\begin{center}
	\lr{$\delta$(q\textsubscript{0}, $\epsilon$, z\textsubscript{0}) → (q\textsubscript{1}, Sz\textsubscript{0})}
\end{center}

سپس برای قانون \lr{ S → aAA} تابع انتقال زیر را انجام می دهیم:

\begin{center}
	\lr{$\delta$(q\textsubscript{1}, a, S) → (q\textsubscript{1}, AA)}
\end{center}

برای قانون \lr{ A → aS} تابع انتقال زیر را داریم:

\begin{center}
	\lr{$\delta$(q\textsubscript{1}, a, A) → (q\textsubscript{1}, S)}
\end{center}

برای قانون \lr{ A → bS} تابع انتقال زیر را داریم:
\begin{center}
	\lr{$\delta$(q\textsubscript{1}, b, S) → (q\textsubscript{1}, S)}
\end{center}

برای قانون \lr{ A → a} تابع انتقال زیر را داریم:
\begin{center}
	\lr{$\delta$(q\textsubscript{1}, a, A) → (q\textsubscript{1}, $\lambda$)}
\end{center}

برای حالت های پذیرش نیزتابع انتقال زیر را خواهیم داشت:
\begin{center}
	\lr{$\delta$(q\textsubscript{1}, $\lambda$, z\textsubscript{0}) → (q\textsubscript{f},z\textsubscript{0})}
\end{center}
که با حالت پایانی پذیرفته می شود

\begin{center}
	\lr{$\delta$(q\textsubscript{1}, $\lambda$, z\textsubscript{0}) → (q\textsubscript{1}, $\lambda$)}
\end{center}
که با حالت پشته خالی پذیرفته می شود



15. برای گرامر زیر یک ماشین پشته ای معادل طراحی کنید:
\begin{center}
	\lr{S → aAA}
\end{center}
\begin{center}
	\lr{A → aS \textbar{} b}
\end{center}

حل: گرامر در فرم نرمال گریباخ می باشد.

ابتدا علامت شروع که همان \lr{ S } است را درون پشته قرار می دهیم با استفاده از قانون زیر

\begin{center}
	\lr{ $\delta$(q\textsubscript{0}, $\epsilon$, z\textsubscript{0}) → (q\textsubscript{1}, Sz\textsubscript{0})}
\end{center}

سپس برای قانون \lr{ S → aAA} تابع انتقال زیر را انجام می دهیم:

\begin{center}
	\lr{$\delta$(q\textsubscript{1}, a, S) → (q\textsubscript{1}, AA)}
\end{center}

برای قانون \lr{ A → aS} تابع انتقال زیر را داریم:

\begin{center}
	\lr{$\delta$(q\textsubscript{1}, a, A) → (q\textsubscript{1}, S)}
\end{center}

برای قانون \lr{ A → b } تابع انتقال زیر را داریم:

\begin{center}
	\lr{$\delta$(q\textsubscript{1}, b, A) → (q\textsubscript{1}, $\lambda$)}
\end{center}

برای حالت های پذیرش نیز تابع انتقال زیر را خواهیم داشت:

\begin{center}
	\lr{$\delta$(q\textsubscript{1}, $\lambda$, z\textsubscript{0}) → (q\textsubscript{f},z\textsubscript{0})}
\end{center}
که با حالت پایانی پذیرفته می شود
\begin{center}
	\lr{$\delta$(q\textsubscript{1}, $\lambda$, z\textsubscript{0}) → (q\textsubscript{1}, $\lambda$)}
\end{center}
که با حالت پشته خالی پذیرفته می شود



16. ماشین پشته ای معادل با گرامر زیر طراحی کنید:

\begin{center}
	\lr{S → aBc}
\end{center}
\begin{center}
	\lr{A → abc}
\end{center}
\begin{center}
	\lr{B → aAb}
\end{center}
\begin{center}
	\lr{C → AB}
\end{center}
\begin{center}
	\lr{C → c}
\end{center}

گرامر در فرم نرمال گریباخ نمی باشد. با جایگزینی \lr{c} با \lr{C} و اضافه کردن قانون \lr{D → b} و جایگزینی \lr{b} با \lr{D} و همچنین جایگزینی \lr{A} از قانون \lr{ C → AB} با \lr{aDC} به فرم نرمال گریباخ تبدیل می شود. گرامر نهایی بصورت زیر می شود:

\begin{center}
	\lr{S → aBC}
\end{center}
\begin{center}
	\lr{A → aDC}
\end{center}
\begin{center}
	\lr{B → aAD}
\end{center}
\begin{center}
	\lr{C → aDCB}
\end{center}
\begin{center}
	\lr{C → c}
\end{center}
\begin{center}
	\lr{D → b}
\end{center}
(حال به ماشین پشته ای معادل تبدیل کنید)

17. ماشین پشته ای

\begin{center}
	\lr{P = (\{p, q\}, \{0, 1\}, (x, z\textsubscript{0}), $\delta$, q,z\textsubscript{0})}
\end{center}

را به فرم نرمال گریباخ تبدیل کنید اگر تابع انتقال بصورت زیر باشد:

\begin{center}
	\lr{$\delta$(q, 1, z\textsubscript{0}) → (q, xz\textsubscript{0})}
\end{center}

حل: ماشین پشته ای معادل دارای دو وضعیت \lr{p} و \lr{q} می باشد. پس دو قانون زیر نیز به گرامر اضافه می شوند:
\begin{center}
	\lr{S → (q z\textsubscript{0} q) \textbar{} (q z\textsubscript{0} p)}
\end{center}

برای تابع انتقال
\lr{$\delta$(q, 1, z\textsubscript{0}) → (q, xz\textsubscript{0})}
قوانین بصورت زیر خواهند بود:
\begin{center}
	\lr{$\delta$(q, 1, z\textsubscript{0}) → (q, xz\textsubscript{0})}
\end{center}
\begin{center}
	\lr{(q z\textsubscript{0} q) → 1(q x q) ( q z\textsubscript{0} q)}
\end{center}
\begin{center}
	\lr{(q z\textsubscript{0} q) → 1(q x p) ( p z\textsubscript{0} q)}
\end{center}
\begin{center}
	\lr{(q z\textsubscript{0} q) → 1(q x q) ( q z\textsubscript{0} p)}
\end{center}
\begin{center}
\lr{(q z\textsubscript{0} q) → 1(q x p) ( p z\textsubscript{0} p)}
\end{center}

\end{document}
