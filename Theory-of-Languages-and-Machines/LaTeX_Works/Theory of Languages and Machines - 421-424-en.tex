\documentclass[]{article}
\usepackage[a4paper,bindingoffset=0.2in,%
left=1in,right=1in,top=1in,bottom=1in,%
footskip=.25in]{geometry}
\usepackage{lmodern}
\usepackage{amssymb,amsmath}
\usepackage{ifxetex,ifluatex}
\usepackage{fixltx2e} % provides \textsubscript
\ifnum 0\ifxetex 1\fi\ifluatex 1\fi=0 % if pdftex
  \usepackage[T1]{fontenc}
  \usepackage[utf8]{inputenc}
\else % if luatex or xelatex
  \ifxetex
    \usepackage{mathspec}
  \else
    \usepackage{fontspec}
  \fi
  \defaultfontfeatures{Ligatures=TeX,Scale=MatchLowercase}
\fi

% use upquote if available, for straight quotes in verbatim environments
\IfFileExists{upquote.sty}{\usepackage{upquote}}{}
% use microtype if available
\IfFileExists{microtype.sty}{%
\usepackage{microtype}
\UseMicrotypeSet[protrusion]{basicmath} % disable protrusion for tt fonts
}{}
\usepackage[unicode=true]{hyperref}
\hypersetup{
            pdfborder={0 0 0},
            breaklinks=true}
\urlstyle{same}  % don't use monospace font for urls
\usepackage{graphicx,grffile}
\makeatletter
\def\maxwidth{\ifdim\Gin@nat@width>\linewidth\linewidth\else\Gin@nat@width\fi}
\def\maxheight{\ifdim\Gin@nat@height>\textheight\textheight\else\Gin@nat@height\fi}
\makeatother
% Scale images if necessary, so that they will not overflow the page
% margins by default, and it is still possible to overwrite the defaults
% using explicit options in \includegraphics[width, height, ...]{}
\setkeys{Gin}{width=\maxwidth,height=\maxheight,keepaspectratio}
\IfFileExists{parskip.sty}{%
\usepackage{parskip}
}{% else
\setlength{\parindent}{0pt}
\setlength{\parskip}{6pt plus 2pt minus 1pt}
}
\setlength{\emergencystretch}{3em}  % prevent overfull lines
\providecommand{\tightlist}{%
  \setlength{\itemsep}{0pt}\setlength{\parskip}{0pt}}
\setcounter{secnumdepth}{0}
% Redefines (sub)paragraphs to behave more like sections
\ifx\paragraph\undefined\else
\let\oldparagraph\paragraph
\renewcommand{\paragraph}[1]{\oldparagraph{#1}\mbox{}}
\fi
\ifx\subparagraph\undefined\else
\let\oldsubparagraph\subparagraph
\renewcommand{\subparagraph}[1]{\oldsubparagraph{#1}\mbox{}}
\fi

\begin{document}

If we have to design the PDA accepted by the empty stack, the  $\lambda$ function is

\begin{center}
$\delta$(q\textsubscript{1}, $\lambda$, z\textsubscript{0}) → (q\textsubscript{1},$\lambda$)
\end{center}

If we have to design the PDA accepted by the final state, the $\lambda$ function is

\begin{center}
$\delta$(q\textsubscript{1}, $\lambda$, z\textsubscript{0}) → (q\textsubscript{f}, $\lambda$)
\end{center}

11. Construct an NPDA that accepts the language generated by the productions S → aSa/bSb/c. Show an instantaneous description of this string abcba for this problem. [WBUT 2007]


Solution: The production rules are not in GNF. So, we need to first convert it into GNF. The production rules are

\begin{center}
S → aSa \textbar{} bSb \textbar{} c
\end{center}

\begin{center}
Let us introduce two new productions  C\textsubscript{a} → a , C\textsubscript{b} → b
\end{center}
The new production rules become
\begin{center}
S → aSC\textsubscript{a}
S → bSC\textsubscript{b}
S → c
C\textsubscript{a} → a
C\textsubscript{b} → b
\end{center}
Now, all the productions are in GNF.
Now, from these productions, a PDA can be easily constructed.
First, the start symbol S is pushed into the stack by the following production

\begin{center}
$\delta$(q\textsubscript{0}, $\epsilon$, z\textsubscript{0}) → (q\textsubscript{1},Sz\textsubscript{0})
\end{center}

For the production S → aSC\textsubscript{a}, the transitional function is


\begin{center}
$\delta$(q\textsubscript{1}, a, S) → (q\textsubscript{1}, SC\textsubscript{a})
\end{center}

For the production S → bSC\textsubscript{b}, the transitional function is
\begin{center}
$\delta$(q\textsubscript{1},b,S) → (q\textsubscript{1}, SC\textsubscript{b})
\end{center}

For the production  S → c, the transitional function is

\begin{center}
$\delta$(q\textsubscript{1}, c, S) → (q\textsubscript{1}, $\lambda$)
\end{center}

For the production   C\textsubscript{a} → a, the transitional function is

\begin{center}
$\delta$(q\textsubscript{1}, a, C\textsubscript{a}) → (q\textsubscript{1},Sz\textsubscript{0})
\end{center}
For the production    C\textsubscript{b} → b, the transitional function is

\begin{center}
$\delta$(q\textsubscript{1}, b, C\textsubscript{b}) → (q\textsubscript{1}, $\lambda$)
\end{center}

For acceptance, the transitional function is
\begin{center}
$\delta$(q\textsubscript{1}, $\lambda$, z\textsubscript{0}) → (q\textsubscript{f},z\textsubscript{0}) // accepted by the final state
\end{center}


\begin{center}
$\delta$(q\textsubscript{1}, $\lambda$, z\textsubscript{0}) → (q\textsubscript{1}, $\lambda$)  // accepted by the empty stack
\end{center}

ID for the String ‘abcba’

\begin{flushleft}
$\delta$(q\textsubscript{0}, \underline{$\epsilon$}abcba, z\textsubscript{0}) → $\delta$(q\textsubscript{1},\emph{a}bcba, Sz\textsubscript{0}) → $\delta$(q\textsubscript{1}, a\underline{b}cba, SC\textsubscript{a}z\textsubscript{0}) → $\delta$(q\textsubscript{1}, ab\underline{c}ba, SC\textsubscript{b}C\textsubscript{a}z\textsubscript{0}) → $\delta$(q\textsubscript{1}, abc\underline{b}a, C\textsubscript{b}C\textsubscript{a}z\textsubscript{0}) → $\delta$(q\textsubscript{1}, abcb\underline{a}, C\textsubscript{a}z\textsubscript{0}) → $\delta$(q\textsubscript{1}, abcbaB, z\textsubscript{0}) → (q\textsubscript{f}, z\textsubscript{0}) (Acceptance by FS).
\end{flushleft}



12. Construct a PDA, A, equivalent to the following context-free grammar

\begin{center}
S → 0BB, B → 0S \textbar{} 1S \textbar{} 0
\end{center}

Test whether 0104 is in N(A).


Solution: The CFG is S → 0BB, B → 0S \textbar{} 1S \textbar{} 0

All the production rules of the grammar are in GNF.
First, the start symbol S is pushed into the stack by the following production

\begin{center}
$\delta$(q\textsubscript{0}, $\epsilon$, z\textsubscript{0}) → (q\textsubscript{1}, Sz\textsubscript{0})
\end{center}

For the production S → 0BB, the transitional function is

\begin{center}
$\delta$(q\textsubscript{1}, 0, S) → (q\textsubscript{1}, BB)
\end{center}

For the production B → 0S, the transitional function is

\begin{center}
$\delta$(q\textsubscript{1}, 0, B) → (q\textsubscript{1}, S)
\end{center}

For the production  B → 1S, the transitional function is

\begin{center}
$\delta$(q\textsubscript{1}, 1, B) → (q\textsubscript{1}, S)
\end{center}

For the production B → 0, the transitional function is

\begin{center}
$\delta$(q\textsubscript{1}, 0, B) → (q\textsubscript{1}, $\lambda$)
\end{center}

For acceptance, the transitional functions are

\begin{center}
$\delta$(q\textsubscript{1}, $\lambda$, z\textsubscript{0}) → (q\textsubscript{f},z\textsubscript{0}) // accepted by the final state
\end{center}

\begin{center}
$\delta$(q\textsubscript{1}, $\lambda$, z\textsubscript{0}) → (q\textsubscript{1}, $\lambda$) // accepted by the empty stack
\end{center}

The ID for the String 010000

\begin{flushleft}
(q\textsubscript{0}, \underline{$\epsilon$}10000, z\textsubscript{0}) → (q\textsubscript{1},\underline{0}10000, Sz\textsubscript{0}) → (q\textsubscript{1}, 0\underline{1}0000, BBz\textsubscript{0}) → (q\textsubscript{1}, 01\underline{0}000, SBz\textsubscript{0}) → (q\textsubscript{1}, 010\underline{0}00, BBBz\textsubscript{0}) → (q\textsubscript{1}, 0100\underline{0}0, BBz\textsubscript{0}) → (q\textsubscript{1}, 01000\underline{0}, Bz\textsubscript{0}) → (q\textsubscript{1}, 010000$\epsilon$, z\textsubscript{0}) → (q\textsubscript{f},010000$\epsilon$, z\textsubscript{0}) (Accepted by the final state).
\end{flushleft}



13. Show that the language L = \{0\textsuperscript{n}1\textsuperscript{n} \textbar{} n \underline{\textgreater{}} 1\} $\bigcup$ \{0\textsuperscript{n}1\textsuperscript{2n} \textbar{} n \underline{\textgreater{}} 1\} is a context-free language that is not accepted by any DPDA. [UPTU 2005]

Solution: The context-free grammar for the language is

\begin{center}
S → S\textsubscript{1} \textbar{} S\textsubscript{2}
\end{center}
\begin{center}
	S\textsubscript{1} → 0S\textsubscript{1}1 \textbar{} 01
\end{center}
\begin{center}
	S\textsubscript{2} → 0S\textsubscript{2}11 \textbar{} 011
\end{center}

The GNF equivalent to the grammar is

\begin{center}
	S → 0S\textsubscript{1}A \textbar{} 0A \textbar{} 0S\textsubscript{2}A \textbar{} 0AA 
\end{center}
\begin{center}
A → 1
\end{center}

The transitional functions of the PDA equivalent to the grammar are

\begin{center}
	$\delta$(q\textsubscript{0}, $\epsilon$, z\textsubscript{0}) → (q\textsubscript{1},Sz\textsubscript{0})
\end{center}
\begin{center}
	$\delta$(q\textsubscript{1}, 0, S) → (q\textsubscript{1}, S\textsubscript{1}A)
\end{center}
\begin{center}
	$\delta$(q\textsubscript{1}, 0, S) → (q\textsubscript{1}, A)
\end{center}
\begin{center}
	$\delta$(q\textsubscript{1}, 0, S) → (q\textsubscript{1}, S\textsubscript{2}A)
\end{center}
\begin{center}
	$\delta$(q\textsubscript{1}, 0, S) → (q\textsubscript{1}, AA)
\end{center}
\begin{center}
	$\delta$(q\textsubscript{1}, 1, A) → (q\textsubscript{1}, $\lambda$)
\end{center}
\begin{center}
	$\delta$(q\textsubscript{1}, $\lambda$, z\textsubscript{0}) → (q\textsubscript{f}, z\textsubscript{0}) // accepted by the final state
\end{center}

\begin{center}
	$\delta$(q\textsubscript{1}, $\lambda$, z\textsubscript{0}) → (q\textsubscript{1}, $\lambda$) // accepted by the empty stack
\end{center}

The PDA is an NPDA, as for the combination (q\textsubscript{1}, 0, S), there are four transitional functions.


14. Convert the CFG into an equivalent PDA. [Cochin University 2006]

\begin{center}
	S → aAA
\end{center}
\begin{center}
	A → aS \textbar{} bS \textbar{} a
\end{center}

Solution: The grammar is in GNF.

First, the start symbol S is pushed into the stack by the following production


\begin{center}
	$\delta$(q\textsubscript{0}, $\epsilon$, z\textsubscript{0}) → (q\textsubscript{1}, Sz\textsubscript{0})
\end{center}

For the production  S → aAA, the transitional function is

\begin{center}
	$\delta$(q\textsubscript{1}, a, S) → (q\textsubscript{1}, AA)
\end{center}

For the production   A → aS, the transitional function is

\begin{center}
	$\delta$(q\textsubscript{1}, a, A) → (q\textsubscript{1}, S)
\end{center}

For the production  A → bS, the transitional function is
\begin{center}
	$\delta$(q\textsubscript{1}, b, S) → (q\textsubscript{1}, S)
\end{center}

For the production  A → a, the transitional function is
\begin{center}
	$\delta$(q\textsubscript{1}, a, A) → (q\textsubscript{1}, $\lambda$)
\end{center}

For acceptance, the transitional functions are
\begin{center}
	$\delta$(q\textsubscript{1}, $\lambda$, z\textsubscript{0}) → (q\textsubscript{f},z\textsubscript{0}) // accepted by the final state
\end{center}

\begin{center}
	$\delta$(q\textsubscript{1}, $\lambda$, z\textsubscript{0}) → (q\textsubscript{1}, $\lambda$) // accepted by the empty stack
\end{center}



15. Construct a PDA equivalent to the grammar
\begin{center}
	S → aAA
\end{center}
\begin{center}
	A → aS \textbar{} b
\end{center}
[Andhra University 2007]
\newline
Solution: The grammar is in GNF.

First, the start symbol S is pushed into the stack by the following production
\begin{center}
	 $\delta$(q\textsubscript{0}, $\epsilon$, z\textsubscript{0}) → (q\textsubscript{1}, Sz\textsubscript{0})
\end{center}

For the production S → aAA, the transitional function is
\begin{center}
	$\delta$(q\textsubscript{1}, a, S) → (q\textsubscript{1}, AA)
\end{center}

For the production  A → aS, the transitional function is

\begin{center}
	$\delta$(q\textsubscript{1}, a, A) → (q\textsubscript{1}, S)
\end{center}

For the production  A → b, the transitional function is

\begin{center}
	$\delta$(q\textsubscript{1}, b, A) → (q\textsubscript{1}, $\lambda$)
\end{center}

For acceptance, the transitional functions are

\begin{center}
	$\delta$(q\textsubscript{1}, $\lambda$, z\textsubscript{0}) → (q\textsubscript{f},z\textsubscript{0}) // accepted by the final state
\end{center}
\begin{center}
	$\delta$(q\textsubscript{1}, $\lambda$, z\textsubscript{0}) → (q\textsubscript{1}, $\lambda$) // accepted by the empty stack
\end{center}



16. Construct a PDA equivalent to the following grammar. [JNTU 2008]

\begin{center}
	S → aBc
\end{center}
\begin{center}
	A → abc
\end{center}
\begin{center}
	B → aAb
\end{center}
\begin{center}
	C → AB
\end{center}
\begin{center}
	C → c
\end{center}


Solution: The grammar is not in GNF. The grammar is converted into GNF by replacing c by C , adding the production D → b, and replacing b by D and replacing A of C → ABby aDC. The final grammar is

\begin{center}
	S → aBC
\end{center}
\begin{center}
	A → aDC
\end{center}
\begin{center}
	B → aAD
\end{center}
\begin{center}
	C →aDCB
\end{center}
\begin{center}
	C → c
\end{center}
\begin{center}
	D → b
\end{center}
(Now convert it to an equivalent PDA.)

17. Convert the PDA 

\begin{center}
	P = (\{p, q\}, \{0, 1\}, (x, z\textsubscript{0}), $\delta$, q,z\textsubscript{0})
\end{center}
to a CFG, if $\lambda$ is given as
\begin{center}
			$\delta$(q, 1, z\textsubscript{0}) → (q, xz\textsubscript{0})       [UPTU 2005]
\end{center}

Solution: The PDA contains two states, p and q. Thus, the following two production rules are added to the grammar. 
\begin{center}
	S → (q z\textsubscript{0} q) \textbar{} (q z\textsubscript{0} p)
\end{center}

For the transitional function  $\delta$(q, 1, z\textsubscript{0}) → (q, xz\textsubscript{0}), the production rules are
\begin{center}
	$\delta$(q, 1, z\textsubscript{0}) → (q, xz\textsubscript{0})
\end{center}
\begin{center}
	(q z\textsubscript{0} q) → 1(q x q) ( q z\textsubscript{0} q)
\end{center}
\begin{center}
	(q z\textsubscript{0} q) → 1(q x p) ( p z\textsubscript{0} q)
\end{center}
\begin{center}
	(q z\textsubscript{0} q) → 1(q x q) ( q z\textsubscript{0} p)
\end{center}
\begin{center}
	(q z\textsubscript{0} q) → 1(q x p) ( p z\textsubscript{0} p)
\end{center}

\end{document}
