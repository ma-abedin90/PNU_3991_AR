%%%%%%%%%%%%%%%%%%%%%%%%%%%%%%%%%%%%%%%%%
% Beamer Presentation
% LaTeX Template
% Version 1.0 (10/11/12)
%
% This template has been downloaded from:
% http://www.LaTeXTemplates.com
%
% License:
% CC BY-NC-SA 3.0 (http://creativecommons.org/licenses/by-nc-sa/3.0/)
%
%%%%%%%%%%%%%%%%%%%%%%%%%%%%%%%%%%%%%%%%%

%----------------------------------------------------------------------------------------
%	PACKAGES AND THEMES
%----------------------------------------------------------------------------------------

\documentclass{beamer}

\mode<presentation> {
	\usetheme{Madrid}
}

\usepackage{graphicx} % Allows including images
\usepackage{booktabs} % Allows the use of \toprule, \midrule and \bottomrule in tables

%----------------------------------------------------------------------------------------
%	TITLE PAGE
%----------------------------------------------------------------------------------------

\title[ma.abedin90@gmail.com]{Problems Project} % The short title appears at the bottom of every slide, the full title is only on the title page

\author{Masoumeh Abedin} % Your name
\institute[PNU] % Your institution as it will appear on the bottom of every slide, may be shorthand to save space
{
	Payam Nour University \\ % Your institution for the title page
	\medskip
	
	\textbf{Theory of Languages and Machine}
	\newline
	\textbf{Dr Ali Razavi}
}
\date{\today} % Date, can be changed to a custom date

\begin{document}
	
	\begin{frame}
		\titlepage % Print the title page as the first slide
	\end{frame}
		
	%----------------------------------------------------------------------------------------
	%	PRESENTATION SLIDES
	%----------------------------------------------------------------------------------------
	
	%------------------------------------------------
	\section{Problem} % Sections can be created in order to organize your presentation into discrete blocks, all sections and subsections are automatically printed in the table of contents as an overview of the talk
	%------------------------------------------------
	
	
	\begin{frame}
		\frametitle{Problem I: Statement}
		Construct an NPDA that accepts the language generated by the productions S → aSa/bSb/c. Show an instantaneous description of this string abcba for this problem. [WBUT 2007]
		
		[Introduction to Automata Theory, Formal Language and Computation, Shyamalendu Kandar - Page 421]
	\end{frame}
	
	%------------------------------------------------
	
	\begin{frame}[fragile] % Need to use the fragile option when verbatim is used in the slide
		\frametitle{Problem I: Solution}

			The production rules are not in GNF.So, we need to first convert it into GNF. The production rules are
			\begin{center}
			S → aSa $\vert$ bSb $\vert$  c
			\end{center}


			Let us introduce two new productions  C\textsubscript{a} → a , C\textsubscript{b} → b
			The new production rules become
			\begin{center}
				S → aSC\textsubscript{a}
			\end{center}
			\begin{center}
				S → bSC\textsubscript{b}
			\end{center}
			\begin{center}
				S → c		
			\end{center}
			\begin{center}
				C\textsubscript{a} → a
			\end{center}
			\begin{center}
				C\textsubscript{b} → b
			\end{center}
	\end{frame}


	%------------------------------------------------

\begin{frame}[fragile] % Need to use the fragile option when verbatim is used in the slide
	\frametitle{Problem I: Solution, Cont.}
	
	Now, all the productions are in GNF. Now, from these productions, a PDA can be easily constructed. First, the start symbol S is pushed into the stack by the following production
	\begin{center}
		$\delta$(q\textsubscript{0}, $\epsilon$, z\textsubscript{0}) → (q\textsubscript{1},Sz\textsubscript{0})
	\end{center}
	
\end{frame}

	%------------------------------------------------

\begin{frame}[fragile] % Need to use the fragile option when verbatim is used in the slide
	\frametitle{Problem I: Solution, Cont.}
	
For the production S → aSC\textsubscript{a}, the transitional function is


\begin{center}
	$\delta$(q\textsubscript{1}, a, S) → (q\textsubscript{1}, SC\textsubscript{a})
\end{center}

For the production S → bSC\textsubscript{b}, the transitional function is
\begin{center}
	$\delta$(q\textsubscript{1},b,S) → (q\textsubscript{1}, SC\textsubscript{b})
\end{center}

For the production  S → c, the transitional function is

\begin{center}
	$\delta$(q\textsubscript{1}, c, S) → (q\textsubscript{1}, $\lambda$)
\end{center}

For the production   C\textsubscript{a} → a, the transitional function is

\begin{center}
	$\delta$(q\textsubscript{1}, a, C\textsubscript{a}) → (q\textsubscript{1},Sz\textsubscript{0})
\end{center}
For the production    C\textsubscript{b} → b, the transitional function is

\begin{center}
	$\delta$(q\textsubscript{1}, b, C\textsubscript{b}) → (q\textsubscript{1}, $\lambda$)
\end{center}

For acceptance, the transitional function is
\begin{center}
	$\delta$(q\textsubscript{1}, $\lambda$, z\textsubscript{0}) → (q\textsubscript{f},z\textsubscript{0}) // accepted by the final state
\end{center}


\begin{center}
	$\delta$(q\textsubscript{1}, $\lambda$, z\textsubscript{0}) → (q\textsubscript{1}, $\lambda$)  // accepted by the empty stack
\end{center}

\end{frame}

	%------------------------------------------------

\begin{frame}[fragile] % Need to use the fragile option when verbatim is used in the slide
	\frametitle{Problem I: Solution, Cont.}
ID for the String ‘abcba’

\begin{flushleft}
	$\delta$(q\textsubscript{0}, \underline{$\epsilon$}abcba, z\textsubscript{0}) → $\delta$(q\textsubscript{1},\emph{a}bcba, Sz\textsubscript{0}) → $\delta$(q\textsubscript{1}, a\underline{b}cba, SC\textsubscript{a}z\textsubscript{0}) → $\delta$(q\textsubscript{1}, ab\underline{c}ba, SC\textsubscript{b}C\textsubscript{a}z\textsubscript{0}) → $\delta$(q\textsubscript{1}, abc\underline{b}a, C\textsubscript{b}C\textsubscript{a}z\textsubscript{0}) → $\delta$(q\textsubscript{1}, abcb\underline{a}, C\textsubscript{a}z\textsubscript{0}) → $\delta$(q\textsubscript{1}, abcbaB, z\textsubscript{0}) → (q\textsubscript{f}, z\textsubscript{0}) (Acceptance by FS).
\end{flushleft}


\end{frame}

	%------------------------------------------------

\begin{frame}[fragile] % Need to use the fragile option when verbatim is used in the slide
	\frametitle{Problem II: Statement}
	
	Construct a PDA, A, equivalent to the following context-free grammar
	
	\begin{center}
		S → 0BB, B → 0S \textbar{} 1S \textbar{} 0
	\end{center}
	
	Test whether 0104 is in N(A).
	
	
	[Introduction to Automata Theory, Formal Language and Computation, Shyamalendu Kandar - Page 422]
	
\end{frame}


	%------------------------------------------------

\begin{frame}[fragile] % Need to use the fragile option when verbatim is used in the slide
	\frametitle{Problem II: Solution}
	
Solution: The CFG is S → 0BB, B → 0S \textbar{} 1S \textbar{} 0

All the production rules of the grammar are in GNF.
First, the start symbol S is pushed into the stack by the following production

	\begin{center}
		$\delta$(q\textsubscript{0}, $\epsilon$, z\textsubscript{0}) → (q\textsubscript{1}, Sz\textsubscript{0})
	\end{center}
\end{frame}





	%------------------------------------------------
	

\begin{frame}[fragile] % Need to use the fragile option when verbatim is used in the slide
	\frametitle{Problem II: Solution, Cont.}

For the production S → 0BB, the transitional function is

\begin{center}
	$\delta$(q\textsubscript{1}, 0, S) → (q\textsubscript{1}, BB)
\end{center}

For the production B → 0S, the transitional function is

\begin{center}
	$\delta$(q\textsubscript{1}, 0, B) → (q\textsubscript{1}, S)
\end{center}

For the production  B → 1S, the transitional function is

\begin{center}
	$\delta$(q\textsubscript{1}, 1, B) → (q\textsubscript{1}, S)
\end{center}

For the production B → 0, the transitional function is

\begin{center}
	$\delta$(q\textsubscript{1}, 0, B) → (q\textsubscript{1}, $\lambda$)
\end{center}

For acceptance, the transitional functions are

\begin{center}
	$\delta$(q\textsubscript{1}, $\lambda$, z\textsubscript{0}) → (q\textsubscript{f},z\textsubscript{0}) // accepted by the final state
\end{center}

\begin{center}
	$\delta$(q\textsubscript{1}, $\lambda$, z\textsubscript{0}) → (q\textsubscript{1}, $\lambda$) // accepted by the empty stack
\end{center}
\end{frame}


	%------------------------------------------------

\begin{frame}[fragile] % Need to use the fragile option when verbatim is used in the slide
	\frametitle{Problem II: Solution, Cont.}
The ID for the String 010000

\begin{flushleft}
	(q\textsubscript{0}, \underline{$\epsilon$}10000, z\textsubscript{0}) → (q\textsubscript{1},\underline{0}10000, Sz\textsubscript{0}) → (q\textsubscript{1}, 0\underline{1}0000, BBz\textsubscript{0}) → (q\textsubscript{1}, 01\underline{0}000, SBz\textsubscript{0}) → (q\textsubscript{1}, 010\underline{0}00, BBBz\textsubscript{0}) → (q\textsubscript{1}, 0100\underline{0}0, BBz\textsubscript{0}) → (q\textsubscript{1}, 01000\underline{0}, Bz\textsubscript{0}) → (q\textsubscript{1}, 010000$\epsilon$, z\textsubscript{0}) → (q\textsubscript{f},010000$\epsilon$, z\textsubscript{0}) (Accepted by the final state).
\end{flushleft}

	
\end{frame}

	%------------------------------------------------
	
	\begin{frame}
		\frametitle{References}
		\footnotesize{
			\begin{thebibliography}{99} % Beamer does not support BibTeX so references must be inserted manually as below
				\bibitem[Book] 
				* An Introduction to Automata Theory, Formal Language and Computation, Shyamalendu Kandar - Pages 421-422
			\end{thebibliography}
		}
	\end{frame}
	
	%------------------------------------------------
	
	\begin{frame}
		\Huge{\centerline{The End}}
	\end{frame}
	
	%----------------------------------------------------------------------------------------
	
\end{document} 